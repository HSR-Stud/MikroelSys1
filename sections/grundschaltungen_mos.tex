\section{Grundschaltungen mit MOS-Transistoren\skript{Kap. 6}}

\subsection{Grundschlatungensarten}
Die Grundschaltungen besitzen je einen Ein- und Ausgang. Derjenige Transistoranschluss, welcher weder Ein- noch Ausgang darstellt gibt der Grundschaltung
den Namen.

\begin{tabular}{|l|l|l|l|}
	\hline
	Grundschaltung		& Eingangsanschluss	& Ausgangsanschluss	& Namensgebender Anschluss
	\\ \hline
	Source Schaltung	& Gate				& Drain				& Source
	\\ \hline
	Gate Schaltung		& Source			& Drain				& Gate
	\\ \hline
	Drain Schaltung	
	(Source-Folger)		& Gate				& Source			& Drain
	\\ \hline
\end{tabular} \\

Eigenschaften:

\begin{tabular}{|l|l|l|}
	\hline
	Grundschaltung & typische Anwendung & Eingangs-/Ausgangswiderstand ($r_{in}$ / $r_{out}$)
	\\ \hline
	Source Schaltung & Verstärker tiefe bis mittlere Frequenzen & gross/gross
	\\ \hline
	Gate Schaltung & Verstärker hohe Frequenzen & klein/gross
	\\ \hline
	Drain Schaltung & Spannungsfolger / Impedanzwandler & gross/klein
	\\ \hline
\end{tabular}