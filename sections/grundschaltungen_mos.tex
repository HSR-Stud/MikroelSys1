\section{Grundschaltungen mit MOS-Transistoren\skript{Kap. 6}}

\subsection{Grundschaltungen}
Die Grundschaltungen besitzen je einen Ein- und Ausgang. Derjenige Transistoranschluss, welcher weder Ein- noch Ausgang darstellt gibt der Grundschaltung
den Namen. \\

\begin{tabularx}{\linewidth}{|X|l|l|l|}
	\hline
	\textbf{Grundschaltung}	& \textbf{Eingangsanschluss} & \textbf{Ausgangsanschluss} & \textbf{Namensgebender Anschluss}
	\\ \hline
	Source Schaltung	& Gate				& Drain				& Source
	\\ \hline
	Gate Schaltung		& Source			& Drain				& Gate
	\\ \hline
	Drain Schaltung (Source-Folger) & Gate				& Source			& Drain 
	\\ \hline
\end{tabularx} \\

\textbf{Eigenschaften:} \\
\begin{tabularx}{\linewidth}{|X|l|l|}
	\hline
	\textbf{Grundschaltung} & \textbf{Typische Anwendung} & \textbf{Eingangs-/Ausgangswiderstand}
	\\ \hline
	Source Schaltung & Verstärker tiefe bis mittlere Frequenzen & gross/gross
	\\ \hline
	Gate Schaltung & Verstärker hohe Frequenzen & klein/gross
	\\ \hline
	Drain Schaltung & Spannungsfolger / Impedanzwandler & gross/klein
	\\ \hline
\end{tabularx}