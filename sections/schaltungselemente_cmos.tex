\section{Schaltungselemente in CMOS\skript{Kap. 4}}

\subsection{Kapazitäten}
Es gibt im wesentlichen drei unterschiedliche Arten von Kapazitäten, welche in einem CMOS-Chip realisiert werden können: 
Poly-Poly-Kapazität ($C'' \approx 1fF/\mu m^2$),
MOS-Kapazität ($C'' \approx 10fF/\mu m^2$),
MIM-Kapazität (Metall-Isolator-Metall, $C'' \approx 1fF/\mu m^2$). $C''$ bezeichnet dabei die spezifische Kapazität pro Flächeneinheit.
Der Plattenabstand $d$ ist meist durch die Herstellung gegeben.

\begin{tabularx}{\linewidth}{|l|X|}
	\hline
	Elektrische Feldkonstante	& $\epsilon_0 = 8.85 \cdot 10^{-12} F/m$
	\\ \hline
	Kapazität/Fläche	& $C'' = \cfrac{\epsilon}{d} = \cfrac{\epsilon_0 \epsilon_r}{d}$
	\\ \hline
	Kapazität & $C = C'' \cdot A$
	\\ \hline
\end{tabularx}

\subsection{Widerstände}
Widerstände in CMOS sind meist unerwünscht da sie viel Platz brauchen und viel Wärme entsteht. Sollte trotzdem ein
Widerstand erforderlich sein, so lässt so ist er über seine Breite und Länge zu definieren:
\[
	R = R_\diamond \frac{L}{W}
\]

\begin{tabularx}{0.8\linewidth}{|l|l|X|l|l|}
	\hline
	\multicolumn{5}{|c|}{\textbf{Typische Werte für Widerstände}}
	\\ \hline
	Poly-Widerstand & $R \approx 10 \Omega/\diamond$ & & HR-Poly-Widerstand & $R \approx 1k\Omega/\diamond$
	\\ \hline
	P-Diffusions-Widerstand & $R \approx 100\Omega/\diamond$ & & N-Diffusions-Widerstand & $R \approx 100\Omega/\diamond$
	\\ \hline
	N-Well-Widerstand & $R \approx 1k\Omega/\diamond$ & & &
	\\ \hline
\end{tabularx}