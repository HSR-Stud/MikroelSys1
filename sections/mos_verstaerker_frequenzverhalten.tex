\section{Frequenzverhalten von MOS Verstärker}
Jeder Knoten N bildet einen Pol bei der Frequenz
$f_N=\frac{1}{2\pi\cdot R_NC_N}$\\
Grobe Analyse: die Knoten mit hohen RC-Produkten suchen. Dort entstehen
Systempole, welche einen Abfall von 20dB/Dekade im Frequenzgang einleiten.
\subsection{Widerstände}
\begin{tabular}{lll}
	Knotenimpedanz praktisch unendlich & $r_{iG}\rightarrow\infty$ & Gate \\
	Knotenimpedanz sehr hoch & $r_{ds}=\frac{1}{g_0}$ & Drain des Transistors wenn
	als Stromquelle beschaltet\\
	Knotenimpedanz tief & $\frac{1}{g_m}$ & Drain des Transistors in
	Diodenschaltung \\
	& & Source des Transistors in Stromquellenschaltung \\
\end{tabular}
\subsection{Kapazitäten}
\begin{tabular}{p{4cm}p{14cm}}
	Knotenkapazität gross & C als passive Schaltungskomponente \\
	Knotenkapazität mittel & parasitäre Kapazität verstärkt durch Miller-Effekt.
	Häufig $C_{GD}$ eines verstärkenden Transistors\\
	Knotenkapazität klein & Knoten mit parasitären Kapazitäten. In der Regel beim
	Gate-Knoten am höchsten.
\end{tabular}
\subsection{Millerkapazität}
Die Miller Kapazität $C_M$ zwischen Ein- und ausgang eines Verstärkers mit
Verstärker A liegt, erscheint:\\
multipliziert mit ($1-A$) parallel zum Eingang ($C_{MI}$)\\
multipliziert mit ($1-\frac{1}{A}$) parallel zum Ausbang ($C_{MO}$)\\
$C_M$ wird aus dem Schema entfernt\\
\subsection{Darstellung}
\begin{enumerate}
  \item DC Verstärkung berechnen (aus Kleinsignalersatzschaltung für
  Niederfrequenz)
  \item Die relevanten Pole finden (An welchem Knoten befindet sich ein hohes
  RC Produkt)
  \item Pole in Bode-Diagramm einzeichnen
\end{enumerate}