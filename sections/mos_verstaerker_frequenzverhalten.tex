\begin{multicols}{2}

\section{Frequenzverhalten von MOS Verstärker}
Jeder Knoten N bildet einen Pol bei der Frequenz
$f_N=\frac{1}{2\pi\cdot R_NC_N}$. \\

\textbf{Grobe Analyse:} Die Knoten mit hohen RC-Produkten suchen. Dort entstehen
Systempole, welche einen Abfall von 20dB/Dekade im Frequenzgang einleiten.

\subsection{Widerstände}
	\textbf{Knotenimpedanz praktisch unendlich:} \\ Gate: $r_{iG}\rightarrow\infty$ \\
	\textbf{Knotenimpedanz sehr hoch:} \\ Drain des Transistors wenn
		als Stromquelle beschaltet: $r_{ds}=\frac{1}{g_0}$  \\	
	\textbf{Knotenimpedanz tief:} \\ Drain des Transistors in
		Diodenschaltung, Source des Transistors in Stromquellenschaltung: $\frac{1}{g_m}$  \\
		
\subsection{Kapazitäten}
	\textbf{Knotenkapazität gross:} \\ C als passive Schaltungskomponente. \\
	\textbf{Knotenkapazität mittel:} \\ parasitäre Kapazität verstärkt durch Miller-Effekt.
	Häufig $C_{GD}$ eines verstärkenden Transistors. \\
	\textbf{Knotenkapazität klein:} \\ Knoten mit parasitären Kapazitäten. In der Regel beim
	Gate-Knoten am höchsten. \\

\columnbreak

\subsection{Millerkapazität}
Die Miller Kapazität $C_M$ zwischen Ein- und Ausgang eines Verstärkers mit
Verstärker A liegt, erscheint:\\

multipliziert mit ($1-A$) parallel zum Eingang ($C_{MI}$)\\
multipliziert mit ($1-\frac{1}{A}$) parallel zum Ausbang ($C_{MO}$)\\

$C_M$ wird aus dem Schema entfernt\\

\subsection{Darstellung}
\begin{enumerate}
  \item DC Verstärkung berechnen (aus Kleinsignalersatzschaltung für
  Niederfrequenz)
  \item Die relevanten Pole finden (An welchem Knoten befindet sich ein hohes
  RC Produkt)
  \item Pole in Bode-Diagramm einzeichnen
\end{enumerate}

\subsection{Typische Kapazitäten}
\adjustbox{max width=\linewidth}{
	\begin{tabular}{|l|l|l|l|l|} 
	\hline
		& $C_{GS}$ & $C_{GD}$ & $C_{SB}$ & $C_{DB}$ \\ 
	\hline
		Gesättigt & $C_{GS0} + \frac{2}{3}C_{oxt}$ & $C_{GD0}$ & $C_{jSBt}+\frac{2}{3}C_{BCt}$ & $C_{jDBt}$ \\
 		Typ. Wert & 33 fF & 1.2 fF & 10 fF & 7 fF \\
	\hline
		Ungesättigt & $C_{GS0}+\frac{1}{2}C_{oxt}$ & $C_{GD0}+\frac{1}{2}C_{oxt}$ & $C_{jSBt}+\frac{1}{2}C_{BCt}$ & $C_{jDBt}+\frac{1}{2}C_{BCT}$ \\
		Typ. Wert & 26 fF & 26 fF & 10 fF & 10 fF \\
	\hline	
	\end{tabular}
} \\

$C_{oxt} = C_{ox} \cdot W \cdot L_{eff}  \quad C_{BCt} = C_{jBC} \cdot W \cdot L_{eff} $ \\
$C_{jSBt} = C_{jSB} \cdot A_S + C_{jswSB} \cdot P_S$ \\
$C_{jDBt} = C_{jDB} \cdot A_D + C_{jswDB} \cdot P_D$ \\

Wenn $V_{SB}=0$, dann $C_{SB}$ ignorieren und $C_{DB}=C_{DS}$.

\end{multicols}