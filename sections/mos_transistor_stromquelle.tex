\section{MOS-Transistor als Stromquelle\skript{Kap. 8}}

{ \centering
\begin{minipage}{0.49 \linewidth}
	\subsection{Strom einer MOS Stromquelle}
	$I_D=\frac{\beta}{2}\left(V_{GS}-V_T\right)^2\left(1+\lambda V_{DS}\right)
	\approx \frac{\beta}{2}\left(V_{GS}-V_T\right)^2 $
\end{minipage}
\begin{minipage}{0.49 \linewidth}
	\subsection{Sättigungsspannung}
	\textbf{Bei starker Inversion}
	$V_{DS,sat}=V_{GS}-V_t=\sqrt{\frac{2I_d}{\beta}}$\\
	\textbf{Bei schwacher Inversion} $V_{DS,sat}\approx 5 \Phi_t \approx 130mV$
\end{minipage}
}

\begin{tabularx}{\linewidth}{|l|p{3cm}|X|X|}
\hline
Schaltung & Konfiguration & Ausgangswiderstand $r_0$ & Minimale Ausgangsspannung $V_{0,min}$ 
\\ \hline
\adjustbox{valign=t, padding=1ex}{\begin{circuitikz}[scale=0.5, transform shape, european]
\ctikzset{tripoles/mos style/arrows}

\draw (1.5,3) node[nmos] (nmos) {}
	(nmos.G) to (0,3) to [V_=VG] (0,0) node[ground] {}
	(nmos.S) to [R=RS] (1.5,0) node[ground] {}
	(nmos.D) to [R=RL,i<_=ID] (1.5,5.5) node[ocirc] {}
;
\draw [->] (2.5,3.5) -- (2.5,0);
\draw node at (2.5,1.75) [anchor=west] {Vo};
\draw node at (1.5,5.5) [anchor=south] {VDD=V+};
\end{circuitikz}} &
Einfache Quelle mit 1 Transistor &
$r_{out}=r_{iD}=r_{DS}=\frac{1}{g_0}=\frac{V_A+V_{DS}}{I_D}\approx\frac{V_A}{I_D}$ &
$V_0 > V_{0,min} = V_{DS,sat}$ 
\\ \hline
 & Stromquelle mit Source-Widerstand &
$r_{iD}=r_{DS}\left(1+\frac{R_S}{r_S}+\frac{R_S}{r_{DS}}\right)=\frac{1}{g_0}\left(1+g_mR_S\right)+R_S$
& $V_0 > V_{0,min} = R_SI_D+V_{DS,sat}$
\\ \hline
\adjustbox{valign=t, padding=1ex}{\begin{circuitikz}[scale=0.5, transform shape, european]
\ctikzset{tripoles/mos style/arrows}

\draw (3,2) node[nmos] (M1) {}
	(3,4) node [nmos] (M2) {}
	(M1.D) -- (M2.S)
	(M2.D) to [R=RL] (3,6.5) node [ocirc] {}
	(M1.S) to [R=RS] (3,0) node [ground] {}
	(M2.G) to (0,4) to [V=VG2] (0,0) node [ground] {}
	(M1.G) to (1.4,2) to [V=VG1] (1.4,0) node [ground] {}
;
\draw (3.2,4.7) -- (5.5,4.7);
\draw (3.2,2.7) -- (4.5,2.7);
\draw node at (4,5.5) {ro2};
\draw [->] (4,5.3) -- (4,4.7);
\draw node at (4,3.5) {ro2};
\draw [->] (4,3.3) -- (4,2.7);
\draw [->] (5.2,4.5) -- (5.2,0);
\draw node at (5.2,2.25) [anchor=west] {Vo};

\end{circuitikz}} & Stromquelle mit Kaskode &
$r_{out}=r_{o2}\approx\frac{r_{DS}^2}{r_{S2}}=\left(\frac{r_{DS}}{r_S}\right)r_{DS}=\mu\cdot
r_{DS}=\frac{1}{g_{o1}}\cdot\frac{g_{m2}}{g_{o2}}$ & 
$V_{0,min}=V_{G2}-V_{GS2}+V_{DS2,sat}$\newline$V_{0,min}=V_{DS1,sat}+V_{DS2,sat}$\newline$
\left(\text{mit } V_{G2}=V_{DS1,sat}+V_{GS2}\right)$
\\ \hline 
\adjustbox{valign=t, padding=1ex}{\begin{circuitikz}[scale=0.5, transform shape, american]
\ctikzset{tripoles/mos style/arrows}

\draw
	(4,1) node [nmos] (M1) {}
	(2,2) node [nmos, scale=-1] (M3) {}
	(4,3) node [nmos] (M2) {}
	
	(M1.S) to (4,0) node [ground] {}
	(M1.G) to (0.5,1) node [ocirc] {}
	(M1.D) to (M2.S)
	
	(M2.G) to (2,3) node [circ] {}
	(M2.D) to (4,4) node [ocirc] {}
	
	(M3.D) to (2,0) node [ground] {}
	(M3.G) to (4,2) node [circ] {}
	(2,4.5) node [ocirc] {} to [I=IQ] (M3.S)
	(0.5,0) node [ground] {}
;
\draw node at (4,1) [anchor=west] {M1};
\draw node at (2,2) [anchor=east] {M2};
\draw node at (4,3) [anchor=west] {M3};
\draw [->] (0.5,1) -- (0.5,0);
\draw node at (0,0.5) {Vi};
\draw [->] (4,5) -- (4,4);
\draw node at (4,4.5) {ID};
\draw [->] (5,4) -- (5,0) {};
\draw node at (5,2) [anchor=west] {Vo};

\end{circuitikz}} & Stromquelle mit
 geregelter Kaskode &$r_{out} \approx
r_{DS1}\cdot\frac{r_{DS2}}{r_{S2}}\cdot\frac{r_{DS3}}{r_{S3}}=\frac{1}{g_{o1}}\cdot\frac{g_{m2}}{g_{o2}}\cdot\frac{g_{m3}}{g_{o3}}
$ &$V_{O,min}=2V_{DS,sat}$
\\ \hline
\end{tabularx}