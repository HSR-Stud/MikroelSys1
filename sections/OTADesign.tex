\section{Designprojekt OTA}
Vorgaben:
\begin{description}[noitemsep, leftmargin=4.2cm, style=sameline]
  \item[Open-Loop-Gain:] $\geq 80 dB$
  \item[Last:] $\leq 5 pF$
  \item[GBW:] $\geq 5 MHz$
  \item[Phase Margin:] $\geq 60^\circ$
  \item[Stabilität:] Unity-Gain stable
  \item[Slew-Rate:] $\geq 20V/\mu S$
  \item[Versorgungsspannung:] $3.3V$
  \item[Output-Swing:] (VSS + 500mV) \ldots (VDD-500mV)
\end{description}

\subsection{Design Ablauf}
OTA wird typischerweise ein Sub-Block einer grösseren Schaltung sein.

\begin{enumerate}[noitemsep]
  \item Definition der Ein- und Ausgägne einer Schaltung (Spezifikation der Schaltung)
  \item Handberechnung, Erstellung des Schaltplanes
  \item Schaltkreissimulation
  \item Erfüllt der Schaltkreis die Spezifikation? Wenn nein, zurück zu 1. Sonst weiter.
  \item Layout
  \item Schaltkreissimulation mit parasitären Einflüssen
  \item Erfüllt der Schaltkreis die Spezifikationen? Nein: Zurück zu Layout, sonst weiter.
  \item Herstellung eines Prototypen
  \item Test und Evaluation
  \item Erfüllt der Schaltkreis die Spezifikation? Nein: Zurück zur Herstellung Prototyp oder zurück zu 1.
  \item Produktion
\end{enumerate}