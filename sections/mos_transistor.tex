\section{MOS-Transistoren\skript{Kap. 5}}

\subsection{Allgemeine Begriffe und Formeln}

\begin{tabular}{|l|l|l|}
	\hline
	$V_T$			& Schwellenspannung		& Threshold Voltage
	\\ \hline
	$V_{DS,sat}$	& Sättigungsspannung	&
	\\ \hline
	$a_A$			& Early-Faktor			&
	\\ \hline
	$\lambda$		& Kanallängen Modulationsfaktor & $\lambda = \frac{1}{V_A}$ 
	\\ \hline
	$V_A$			& Early-Spannung		& $V_A \approx a_A \cdot L$
	\\ \hline
	$\Phi_t$		& Temperaturspannung	& $\Phi_t = V_{temp} = \frac{kT}{e} \quad k = 1.38 \cdot 10^{23} J/K \quad e=1.6 \cdot 10^{-19} C$
	\\ \hline
	$r_{DS}$		& Ausgangswiderstand	& $r_{DS} = \frac{1}{g_0} \approx \frac{\Delta V_{DS}}{\Delta I_D} \quad 
											  \text{oder} \quad r_{DS} = \frac{V_A + V_{DS}}{I_{D,real}} \approx \frac{V_A}{I_D} $
	\\ \hline
	
\end{tabular}

\subsection{Inversion}

\begin{tabular}{|l|l|l|}
	\hline
	\textbf{Arbeitsbereich}	& \textbf{Bedingung}					& \textbf{Sättigungspannung}
	\\ \hline
	weak inversion			& $0 < V_{GS} < V_T - 60mV$				& $V_{DS,sat} \approx 5\Phi_t \approx 130mV \quad \text{(bei } T = 300K \text{)}$
	\\ \hline
	moderate inversion		& $V_T - 60mV < V_{GS} < V_T + 160mV$	&
	\\ \hline
	strong inversion		& $V_T + 160mV < V_{GS} $				& $V_{DS,sat} = V_{GS} - V_T$
	\\ \hline
\end{tabular}


\subsection{Betriebsarten}
\subsubsection{Ungesättigter Betrieb}
Bei $V_{DS} = 0$ verhält sich der Kanal wie ein Widerstand. Die Kennlinie ist eine Gerade durch den Ursprung. Je steiler diese Gerade ist, desto
kleiner ist der Widerstand $r_{DS}$.

\subsubsection{Gesättigter Betrieb}
Bei $V_{DS} \geq V_{DS,sat}$ verhält sich der Kanal wie eine Stromquelle. Wenn die Geraden horizontal verlaufen ist $r_{DS} = \infty$ und
es handelt sich um einen idealen Transistor. Beim realen Transistor steigen jedoch diese Geraden immer leicht an.
Der Anstieg entspricht dem \textbf{Ausgangsleitwert} $g_0$, respektive dem \textbf{Ausgangswiderstand} $r_{DS}$.
Andere Bezeichnungen: differentieller- , Kleinsignal-, dynamischer Ausgangswiderstand.
\[
	r_{DS} = \frac{1}{g_0} = \frac{dV_{DS}}{dI_{D}} \approx \frac{\Delta V_{DS}}{\Delta I_D}
\]


\subsection{Transferkennlinie}

TODO: Abbildung wie Skript S.44 schweissen.

\subsection{Drainstromgleichungen}
\begin{tabular}{|l|l|l|}
	\hline
		\textbf{Ausgangsstrom} 
		& \multicolumn{2}{c|}{\textbf{Ausgangsspannungsbereich} ($V_{DS}$-Bereich)}
	\\
		($I_D-, V_{GS}$-Bereich)
		& Transistor ungesättigt ($V_{DS} < V_{DS,sat}$)
		& Transistor gesättigt ($V_{DS} > V_{DS,sat}$)
	\\ \hline
		\multicolumn{3}{|c|}{\textbf{n-Kanal ohne Kanallängenmodulation und ohne unterschiedlicher Transkonduktanz:}}
	\\ \hline
		EXP-Bereich
		& $I_D = I_M e^{\frac{V_{GS}-V_M}{n_M \Phi_t}} (1-e^{\frac{-V_{DS}}{\Phi_t}})$
		& $I_D = I_M e^{\frac{V_{GS}-V_M}{n_M \Phi_t}}$
	\\ \hline
		QUAD-Bereich
		& $I_D = \beta [(V_{GS} - V_T) V_{DS} - \frac{V_{DS}^2}{2}]$
		& $I_D = \frac{\beta}{2}(V_{GS} - V_T)^2$
	\\ \hline
		\multicolumn{3}{|c|}{\textbf{n-Kanal mit Kanallängenmodulation und mit unterschiedlicher Transkonduktanz:}}
	\\ \hline
		EXP-Bereich
		& $I_D = I_M e^{\frac{V_{GS}-V_M}{n_M \Phi_t}} (1-e^{\frac{-V_{DS}}{\Phi_t}}) (1 + \lambda V_{DS})$
		& $I_D = I_M e^{\frac{V_{GS}-V_M}{n_M \Phi_t}} (1 + \lambda V_{DS})$
	\\ \hline
		QUAD-Bereich
		& $I_D = B [(V_{GS} - V_T) V_{DS} - \frac{V_{DS}^2}{2}] (1 + \lambda V_{DS})$
		& $I_D = \frac{\beta}{2}(V_{GS} - V_T)^2 (1 + \lambda V_{DS})$
	\\ \hline
		\multicolumn{3}{|c|}{\textbf{p-Kanal mit Kanallängenmodulation und mit unterschiedlicher Transkonduktanz:}}
	\\ \hline
		EXP-Bereich
		& $I_D = I_M e^{-\frac{V_{GS}-V_M}{n_M \Phi_t}} (1-e^{\frac{-V_{DS}}{\Phi_t}}) (1 - \lambda V_{DS})$
		& $I_D = I_M e^{-\frac{V_{GS}-V_M}{n_M \Phi_t}} (1 - \lambda V_{DS})$
	\\ \hline
		QUAD-Bereich
		& $I_D = -B [(V_{GS} - V_T) V_{DS} - \frac{V_{DS}^2}{2}] (1 - \lambda V_{DS})$
		& $I_D = -\frac{\beta}{2}(V_{GS} - V_T)^2 (1 - \lambda V_{DS})$
	\\ \hline
\end{tabular}
\newline

Die Sättigungsspannung beträgt im \textbf{EXP-Bereich} $5\Phi_t$, im \textbf{QUAD-Bereich} $V_{GS}-V_T$.

Parameter der Gleichungen:

\begin{tabularx}{\linewidth}{|l|l|X|}
	\hline
		$V_T$ & Schwellenspannung &
		Typisch $0.6V$ beim n-Kanal, resp. $-0.6V$ beim p-Kanal. $V_T$ ist stark von der Source-Bulk-Spannung abhängig (Body-Effekt):
		\[ 
			V_T = V_{T0} \pm \Delta V_T \quad \text{mit} \quad \Delta V_T = \gamma(\sqrt{V_SB \pm \Phi_0} -\sqrt{\Phi_0})
		\]
		positives Vorzeichen für n-Kanal, negatives für p-Kanal, $\gamma_N \approx 0.6\sqrt{V}$, $\gamma_P \approx 0.5\sqrt{V}$ 
	\\ \hline
		$\Phi_t$ & Temperaturspannung &
		\[
			\Phi_t = V_{Temp} = \frac{kT}{e} = 86.2 \frac{\mu V}{K}T
		\]
		somit ist $\Phi_t = 25.9mV$ bei $T=300^\circ K$ bzw. $27^\circ C$
	\\ \hline
		$I_M$ & Drainstrom &
		Drainstrom an der Grenze zwischen schwacher und moderater Inversion.
		\[
			I_M = \frac{W}{L} \cdot I_M'
		\]
		$I_M'$ ist der spezifische Drainstrom an der Grenze
	\\ \hline
		$n_M$ & Unterschwellen-Neigungsfaktor &
		Der Faktor $n_m$ ist von der Source-Bulk-Spannung $V_{SB}$ abhängig:
		\[
			n_M = 1 + \frac{\gamma}{2 \sqrt{V_{SB} + \Phi_0}}
		\]
		Für $V_{SB} = 0$ erhalten wir $n_M=1.39$. Häufig wird ein Wert von $n_M \approx 1.5$ angegeben.
	\\ \hline
		$\lambda$ & Kanallängen-Modulationsfaktor &
		inverser Wert der Early-Spannung
		\[
			\lambda = \frac{1}{V_A} \approx \frac{1}{a_A L}
		\]
		Der MOS-Transistor wird vielfach mit $\lambda = 0$ idealisiert, was die Handrechnung vereinfacht.
	\\ \hline
		$B, \beta$ & Transkondukdanz &
		Steilheit, Verstärkungsfaktor. Dieser Faktor ist im gesättigten und ungesättigten Betrieb \textbf{grundsätzlich verschieden}.
	\\ \hline
\end{tabularx}