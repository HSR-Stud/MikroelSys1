\section{MOS-Transistoren\skript{Kap. 5}}

\subsection{Allgemeine Begriffe und Formeln}

\begin{tabular}{|l|l|l|}
	\hline
	$V_T$			& Schwellenspannung		& Threshold Voltage
	\\ \hline
	$V_{DS,sat}$	& Sättigungsspannung	&
	\\ \hline
	$V_A$			& Early-Spannung		&
	\\ \hline
	$\Phi_t$		& Temperaturspannung	& $\Phi_t = V_{temp} = \frac{kT}{e} \quad k = 1.38 \cdot 10^{23} J/K \quad e=1.6 \cdot 10^{-19} C$
	\\ \hline
	$r_{DS}$		& Ausgangswiderstand	& $r_{DS} = \frac{1}{g_0} \approx \frac{\Delta V_{DS}}{\Delta I_D} \quad 
											  \text{oder} \quad r_{DS} = \frac{V_A + V_{DS}}{I_{D,real}} \approx \frac{V_A}{I_D} $
	\\ \hline
	
\end{tabular}

\subsection{Inversion}

\begin{tabular}{|l|l|l|}
	\hline
	\textbf{Arbeitsbereich}	& \textbf{Bedingung}					& \textbf{Sättigungspannung}
	\\ \hline
	weak inversion			& $0 < V_{GS} < V_T - 60mV$				& $V_{DS,sat} \approx 5\Phi_t \approx 130mV \quad \text{(bei } T = 300K \text{)}$
	\\ \hline
	moderate inversion		& $V_T - 60mV < V_{GS} < V_T + 160mV$	&
	\\ \hline
	strong inversion		& $V_T + 160mV < V_{GS} $				& $V_{DS,sat} = V_{GS} - V_T$
	\\ \hline
\end{tabular}


\subsection{Betriebsarten}
\subsubsection{Ungesättigter Betrieb}
Bei $V_{DS} = 0$ verhält sich der Kanal wie ein Widerstand. Die Kennlinie ist eine Gerade durch den Ursprung. Je steiler diese Gerade ist, desto
kleiner ist der Widerstand $r_{DS}$.

\subsubsection{Gesättigter Betrieb}
Bei $V_{DS} \geq V_{DS,sat}$ verhält sich der Kanal wie eine Stromquelle. Wenn die Geraden horizontal verlaufen ist $r_{DS} = \infty$ und
es handelt sich um einen idealen Transistor. Beim realen Transistor steigen jedoch diese Geraden immer leicht an.
Der Anstieg entspricht dem \textbf{Ausgangsleitwert} $g_0$, respektive dem \textbf{Ausgangswiderstand} $r_{DS}$.
Andere Bezeichnungen: differentieller- , Kleinsignal-, dynamischer Ausgangswiderstand.
\[
	r_{DS} = \frac{1}{g_0} = \frac{dV_{DS}}{dI_{D}} \approx \frac{\Delta V_{DS}}{\Delta I_D}
\]

