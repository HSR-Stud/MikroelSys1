\section{MOS Operationsverstärker}

\subsection{Struktur}
\begin{tabular}{p{3cm}p{15cm}}
	Differenzstufe (Eingangsstufe) & bildet die Differenz der Eingangssignale,
	verstärkt sie mit dem Differenz-Verstärkungsfaktor\\
	Verstärkungsstufe	(Integratorstufe) & Verstärkerstufe, erhöht die
	Gesamtverstärkung. Bestimmt meist die Gesamtbandbreite des
	Operationsverstärkers\\
	Leistungsstufe (Ausgangsstufe) & Impedanzwandler. Verstärkung ist selten
	grösser als eins. Verkleinert den Ausgangswiderstand und stellt genügend
	Ausgansstrom zur verfügung\\
\end{tabular}

\subsection{Die wichtigsten Formeln}
\begin{tabular}{p{7cm}p{11cm}}
Verstärkung Differenzstufe &
\textbf{Bei Widerstandslast} $i_{out}=-\frac{g_mv_d}{2}$
$a\approx\frac{g_m\cdot r_{out}}{2}$ \newline
\textbf{Bei Stromspiegellast} $i_{out}=-g_mv_d$ $a\approx g_m\cdot r_{out}$
\\
Grenzwert bei starker Inversion & $a=V_A\sqrt{\frac{\beta}{I_Q}}$ (Bedingung:
$a_{E_N}=a_{E_P}$)\\
Grenzwert bei schwacher Inversion & $a=\frac{V_A}{2n_M\Phi_t}$\\
Gain-Bandwith-Product & $GBP=|a|\cdot f_{P1}$ \\
Slew-Rate & $SR=\frac{dv_o}{dt}=\frac{I_{out}}{C_L}$ \\
Common mode rejection ratio & $CMRR=\left| \frac{a_{DM}}{a_{CM}}\right|$\\
Power supply rejection ratio &
$PSSR_+ = \left| \frac{a_{DM}}{a_{PS+}}\right|$ \newline
$PSSR_- = \left| \frac{a_{DM}}{a_{PS-}}\right|$ \\
Offset: Designregel & Symmetrie, gleiche Stromdichten $\frac{I}{W/L}$ in
Stromspiegeltransistoren\\
\end{tabular}
