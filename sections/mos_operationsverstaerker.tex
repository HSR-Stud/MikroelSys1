\section{MOS Operationsverstärker}

\subsection{Struktur}
\begin{tabular}{p{3cm}p{15cm}}
	Differenzstufe (Eingangsstufe) & bildet die Differenz der Eingangssignale,
	verstärkt sie mit dem Differenz-Verstärkungsfaktor\\
	Verstärkungsstufe	(Integratorstufe) & Verstärkerstufe, erhöht die
	Gesamtverstärkung. Bestimmt meist die Gesamtbandbreite des
	Operationsverstärkers\\
	Leistungsstufe (Ausgangsstufe) & Impedanzwandler. Verstärkung ist selten
	grösser als eins. Verkleinert den Ausgangswiderstand und stellt genügend
	Ausgansstrom zur Verfügung\\
\end{tabular}

\subsection{Schaltungsteile für MOS Operationsverstärker}

\begin{figure}[H]
	\centering
	\begin{subfigure}[b]{4cm}
		\centering
		{\begin{circuitikz}[american, european resistors, scale=0.5, transform shape]
\ctikzset{tripoles/mos style/arrows}

\draw
	(2,2) node [nmos] (M1) {}
	(4,2) node [nmos, xscale=-1] (M2) {}
	
	(M1.S) to (M2.S)
	(M1.G) to (0.5,2) node [ocirc] {Vi1}
	(2,4.5) to [R=RD1,i=ID1] (M1.D)
	
	(M2.G) to (5.5,2) to (5.5,1) to (0.5,1) node [ocirc] {Vi2}
	(4,4.5) node [circ] {} to [R=RD2,i=ID2=Io] (M2.D)
	
	(5.5,4.5) node [ocirc] {V+} to (2,4.5)
	(3,1.25) node [circ] {} to [I] (3,-0.5) to (5.5,-0.5) node [ocirc] {V-}
;

\end{circuitikz}}
		\caption{Differenzstufe für Widerstandslast}
	\end{subfigure}\qquad
	\begin{subfigure}[b]{4cm}
		\centering
		{\begin{circuitikz}[american, european resistors, scale=0.5, transform shape]
\ctikzset{tripoles/mos style/arrows}

\draw
	(2,2) node [nmos] (M1) {}
	(4,2) node [nmos, xscale=-1] (M2) {}
	
	(M1.S) to (M2.S)
	(M1.G) to (0.5,2) node [ocirc] {Vi1}
	(2,3) to (M1.D)
	
	(M2.G) to (5.5,2) to (5.5,1) to (0.5,1) node [ocirc] {Vi2}
	(4,3) to (M2.D)
	
	(5.5,4.5) node [ocirc] {V+} to (3,4.5) to (3,4)
	(1,4) rectangle (5,3)
	(3,1.25) node [circ] {} to [I=$I_Q$] (3,-0.5) to (5.5,-0.5) node [ocirc] {V-}
	(3,3.5) node {1:1}
	(4,2.7) to [short, i<_=iout] (6,2.7) node [ocirc] {}
;

\end{circuitikz}}
		\caption{Differenzstufe für Stromspiegellast}
	\end{subfigure}
	\qquad
	\begin{subfigure}[b]{4cm}
		\centering
		{\begin{circuitikz}[american, scale=0.5, transform shape]
\ctikzset{tripoles/mos style/arrows}

\draw
	(2,1) node [nmos] (M1) {}
	
	(2,4) node [circ] {} to [I=$I_B$] (M1.D)
	(0,1) node [ocirc] {} to [short, i=$I_I$] (0.5,1) to (M1.G)
	(M1.S) to (2,0) node [ground] {}
	
	(0.5,1) to (0.5,2) to [C=$C_1$] (2,2) to [short, i<=$I_O$] (4,2) node [ocirc] {}
	
	(0,0) node [ground] {}
	(4,0) node [ground] {}
	
;

\draw [->] (4,1.8) -- (4,0.2) node [anchor=south west] {$V_O$};
\draw [->] (0,0.8) -- (0,0.2) node [anchor=south west] {$V_I$};


\end{circuitikz}}
		\caption{Verstärkerstufe}
	\end{subfigure}
	\qquad
	\begin{subfigure}[b]{4cm}
		\centering
		{\begin{circuitikz}[american, scale=0.5, transform shape]
\ctikzset{tripoles/mos style/arrows}

\draw
	(0,0) node [ground] {}
	(2,0) node [ground] {}
	(4,0) node [ground] {}
	(2,3) node [nmos] (M1) {}
	
	(M1.S) to [I=$I_B$]  (2,0)
	(0,3) node [ocirc] {} to [short, i=$I_1$] (M1.G)
	(2,4) node [ocirc] {} to (M1.D)
	(4,2) node [ocirc] {} to [short, i=$I_O$] (2,2) node [circ] {}
;
\draw [->] (4,1.8) -- (4,0.2) node [anchor=south west] {$V_O$};
\draw [->] (0,2.8) -- (0,0.2) node [anchor=south west] {$V_I$};


\end{circuitikz}}
		\caption{Leistungsstufe}
	\end{subfigure}
	\label{fig:Schaltungsteile}
\end{figure}

\subsection{Differenz-Stufe}
Eigenschaften der Differenz-Stufe bei strong inversion:
\begin{table}[htbp]
	\centering
	\begin{tabularx}{0.7\linewidth}{lX}
		$V_d \leq \pm \frac{1}{2} \sqrt{\frac{I_Q}{\beta}}$ & Linearität der Diff-Stufe gut. An Linearitätsgrenzen fliesst $I_D = 0.74 \, I_Q$ bzw. $I_D = 0.26 \, I_Q$ \\
		$V_d = \pm \sqrt{\frac{I_Q}{\beta}}$ & In einem der Zweige fliesst $I_D = 0.93 \, I_Q$, im anderen \newline $I_D = 0.07 \, I_Q$. \\
		$V_d = \pm \sqrt{2} \sqrt{\frac{I_Q}{\beta}}$ & Der gesamte $I_Q$ fliesst in einem der beiden Zweige. \\
	\end{tabularx}
\end{table}

\subsection{Die wichtigsten Formeln}
\begin{tabular}{p{7cm}p{11cm}}
Verstärkung Differenzstufe &
\textbf{Bei Widerstandslast} $i_{out}=-\frac{g_mv_d}{2} \quad a\approx\frac{g_m\cdot r_{out}}{2}$ \newline
\textbf{Bei Stromspiegellast} $i_{out}=-g_mv_d \quad a\approx g_m\cdot r_{out}$
\\
Grenzwert bei starker Inversion & $a=V_A\sqrt{\frac{\beta}{I_Q}}$ (Bedingung:
$a_{E_N}=a_{E_P}$)\\
Grenzwert bei schwacher Inversion & $a=\frac{V_A}{2n_M\Phi_t}$\\
Gain-Bandwith-Product & $GBP=|a|\cdot f_{P1} = -\frac{g_m}{2\pi C_L}$ \\
Common mode rejection ratio & $CMRR=\left| \frac{a_{DM}}{a_{CM}}\right| = \frac{r_q}{r_s} = \frac{g_m}{g_{0b}}$ mit $r_q$: Innenwid. von $I_q$ \\
Power supply rejection ratio &
$PSSR_+ = \left| \frac{a_{DM}}{a_{PS+}}\right|$ \newline
$PSSR_- = \left| \frac{a_{DM}}{a_{PS-}}\right|$ \\
Offset: Designregel & Symmetrie, gleiche Stromdichten $\frac{I}{W/L}$ in
allen Stromspiegeltransistoren. \\
\end{tabular}

%TODO: newpage entfernen
\newpage

\subsection{Slew-Rate}
\begin{multicols}{2}
\begin{tabularx}{0.95\linewidth}{Xl}
	Definition Slew-Rate & $SR = \frac{d v_o}{dt} = \frac{I_{out}}{C_L}$ \\
	Differenzstufe & $SR_r = |SR_f| = \frac{I_Q}{C_L}$ \\
	Mit Verstärkung nach SR-dominanter Stufe & $|SR| = \frac{d v_{CL}}{dt} \cdot a$ \\
\end{tabularx}

\columnbreak

\textbf{Vorgehen bei mehrstufigem Verstärker:}
\begin{enumerate}
	\item SR jeder einzelnen Verstärkerstufe untersuchen
	\item SR auf den Ausgang beziehen ($\cdot a$)
	\item Verstärkerstufe mit kleinster SR ist dominant.
\end{enumerate}
\end{multicols}