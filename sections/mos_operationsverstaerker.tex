\section{MOS Operationsverstärker}

\subsection{Struktur}
\begin{tabular}{p{3cm}p{15cm}}
	Differenzstufe (Eingangsstufe) & bildet die Differenz der Eingangssignale,
	verstärkt sie mit dem Differenz-Verstärkungsfaktor\\
	Verstärkungsstufe	(Integratorstufe) & Verstärkerstufe, erhöht die
	Gesamtverstärkung. Bestimmt meist die Gesamtbandbreite des
	Operationsverstärkers\\
	Leistungsstufe (Ausgangsstufe) & Impedanzwandler. Verstärkung ist selten
	grösser als eins. Verkleinert den Ausgangswiderstand und stellt genügend
	Ausgansstrom zur verfügung\\
\end{tabular}

\subsection{Differenzstufe}
\begin{figure}[h]
	\centering
	\begin{subfigure}[b]{5cm}
		\centering
		{\begin{circuitikz}[american, european resistors, scale=0.5, transform shape]
\ctikzset{tripoles/mos style/arrows}

\draw
	(2,2) node [nmos] (M1) {}
	(4,2) node [nmos, xscale=-1] (M2) {}
	
	(M1.S) to (M2.S)
	(M1.G) to (0.5,2) node [ocirc] {Vi1}
	(2,4.5) to [R=RD1,i=ID1] (M1.D)
	
	(M2.G) to (5.5,2) to (5.5,1) to (0.5,1) node [ocirc] {Vi2}
	(4,4.5) node [circ] {} to [R=RD2,i=ID2=Io] (M2.D)
	
	(5.5,4.5) node [ocirc] {V+} to (2,4.5)
	(3,1.25) node [circ] {} to [I] (3,-0.5) to (5.5,-0.5) node [ocirc] {V-}
;

\end{circuitikz}}
		\caption{Widerstandslast}
	\end{subfigure}\qquad
	\begin{subfigure}[b]{5cm}
		\centering
		{\begin{circuitikz}[american, european resistors, scale=0.5, transform shape]
\ctikzset{tripoles/mos style/arrows}

\draw
	(2,2) node [nmos] (M1) {}
	(4,2) node [nmos, xscale=-1] (M2) {}
	
	(M1.S) to (M2.S)
	(M1.G) to (0.5,2) node [ocirc] {Vi1}
	(2,3) to (M1.D)
	
	(M2.G) to (5.5,2) to (5.5,1) to (0.5,1) node [ocirc] {Vi2}
	(4,3) to (M2.D)
	
	(5.5,4.5) node [ocirc] {V+} to (3,4.5) to (3,4)
	(1,4) rectangle (5,3)
	(3,1.25) node [circ] {} to [I=$I_Q$] (3,-0.5) to (5.5,-0.5) node [ocirc] {V-}
	(3,3.5) node {1:1}
	(4,2.7) to [short, i<_=iout] (6,2.7) node [ocirc] {}
;

\end{circuitikz}}
		\caption{Stromspiegellast}
	\end{subfigure}
	\caption{Differenzstufen}
	\label{fig:Differenzstufen}
\end{figure}\\

\subsection{Die wichtigsten Formeln}
\begin{tabular}{p{7cm}p{11cm}}
Verstärkung Differenzstufe &
\textbf{Bei Widerstandslast} $i_{out}=-\frac{g_mv_d}{2}$
$a\approx\frac{g_m\cdot r_{out}}{2}$ \newline
\textbf{Bei Stromspiegellast} $i_{out}=-g_mv_d$ $a\approx g_m\cdot r_{out}$
\\
Grenzwert bei starker Inversion & $a=V_A\sqrt{\frac{\beta}{I_Q}}$ (Bedingung:
$a_{E_N}=a_{E_P}$)\\
Grenzwert bei schwacher Inversion & $a=\frac{V_A}{2n_M\Phi_t}$\\
Gain-Bandwith-Product & $GBP=|a|\cdot f_{P1}$ \\
Slew-Rate & $SR=\frac{dv_o}{dt}=\frac{I_{out}}{C_L}$ \\
Common mode rejection ratio & $CMRR=\left| \frac{a_{DM}}{a_{CM}}\right|$\\
Power supply rejection ratio &
$PSSR_+ = \left| \frac{a_{DM}}{a_{PS+}}\right|$ \newline
$PSSR_- = \left| \frac{a_{DM}}{a_{PS-}}\right|$ \\
Offset: Designregel & Symmetrie, gleiche Stromdichten $\frac{I}{W/L}$ in
Stromspiegeltransistoren\\
\end{tabular}
