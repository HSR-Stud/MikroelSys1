\section{Spannungsreferenzen}
\begin{figure}[!h]
	\centering
	\begin{subfigure}[b]{10cm}
		\centering
		{\begin{circuitikz}[scale=1.5]

\draw
	(0,0) node [ground] {}
	to [V<=$\Phi_t$] (0,1)
	to (1,1)
	(1,1.5) rectangle (2,0.5)
	(1.5,1) node {K}
	(2,1) to (3,1) to (3,1.5)
	(0,2) node [ground] {}
	to [V<=$V_D$] (0,3)
	to (3,3) to (3,2.5)
	(3,2) node {$+$} circle (0.5)
	(3.5,2) to [short] (4,2)
;

\end{circuitikz}}
		\caption{Prinzip der Spannungsreferenz}
	\end{subfigure}\qquad
	\begin{subfigure}[b]{8cm}
		\centering
		{\begin{circuitikz}[scale=1.5, european]

\draw
	(0,0) node [ground] {}
	(1,0) node [ground] {}
	(2,0) node [ground] {}
	(3,0) node [ground] {}
	(4,0) node [ground] {}
	
	(0,1) node [pnp, xscale=-1] (T1) {}
	(3,1) node [pnp] (T2) {}

	(T1.B) -| (1,0)
	(T1.C) -- (0,0)
	(T1.E) -- (0,3)
	
	(T2.B) -| (2,0)
	(T2.C) -- (3,0)
	(T2.E) to [R=$R_3$] (3,3)
	
	(1.5,4) node [op amp, rotate=90, yscale=-1] (opamp) {}
	
	(0,3) -| (opamp.+)
	(3,3) -| (opamp.-)
	
	(0,3) to [R=$R_1$, *-] (0,5) to [short, -o] (4,5)
	(3,3) to [R=$R_2$, *-*] (3,5)
	
	(opamp.out) to [short, -*] (1.5,5)
	
	(4,4.5) to (4,0.5)
	(4,2.5) node [anchor=east] {$V_{ref}$}
		
	(0.5,1.5) -- (2.5,1.5)
	
	(1.5,1.5) node [anchor=south] {$\Delta V_D$}
;

\end{circuitikz}}
		\caption{Praktische Schaltung}
	\end{subfigure}
	\caption{Bandgap-Schaltungen}
	\label{fig:spannungsreferenzen}
\end{figure}

$V_{ref}=V_D+K\cdot \Phi_t$\\
$\Phi_t = \frac{kT}{e}$, ist bei Raumtemperatur $27^\circ C$ $\Phi_t=25.9mV$\\
\begin{tabular}{ll}
Boltzmann-Konstante & $k=1.38\cdot 10^{-23}\frac{J}{K}$\\
absolute Temperatur & T = Temperatur in Kelvin\\
Elementarladung & $e=1.60\cdot 10^{-19}C$\\
\end{tabular}\\
\textbf{Realisierung:}\\
Die beiden Emitterflächen werden mit $A_1$ bzw. $A_2$ bezeichnet.\\
$V_{ref}=V_{EB1}+\Phi_t\cdot\frac{R_2}{R_3}\cdot\ln\left(\frac{R_2}{R_1}\cdot\frac{A_2}{A_1}\right)$